%%%%%%%%%%%%%%%%%%%%%%%%%%%%%%%%%%%%%%%%%
% kaobook
% LaTeX Template
% Version 1.3 (December 9, 2021)
%
% This template originates from:
% https://www.LaTeXTemplates.com
%
% For the latest template development version and to make contributions:
% https://github.com/fmarotta/kaobook
%
% Authors:
% Federico Marotta (federicomarotta@mail.com)
% Based on the doctoral thesis of Ken Arroyo Ohori (https://3d.bk.tudelft.nl/ken/en)
% and on the Tufte-LaTeX class.
% Modified for LaTeX Templates by Vel (vel@latextemplates.com)
%
% License:
% CC0 1.0 Universal (see included MANIFEST.md file)
%
%%%%%%%%%%%%%%%%%%%%%%%%%%%%%%%%%%%%%%%%%

%----------------------------------------------------------------------------------------
%  PACKAGES AND OTHER DOCUMENT CONFIGURATIONS
%----------------------------------------------------------------------------------------

\documentclass[
  a4paper, % Page size
  fontsize=10pt, % Base font size
  twoside=true, % Use different layouts for even and odd pages (in particular, if twoside=true, the margin column will be always on the outside)
  %open=any, % If twoside=true, uncomment this to force new chapters to start on any page, not only on right (odd) pages
  %chapterentrydots=true, % Uncomment to output dots from the chapter name to the page number in the table of contents
  numbers=noenddot, % Comment to output dots after chapter numbers; the most common values for this option are: enddot, noenddot and auto (see the KOMAScript documentation for an in-depth explanation)
]{kaobook}

% Choose the language
\ifxetexorluatex
  \usepackage{polyglossia}
  \setmainlanguage{english}
\else
  \usepackage[english]{babel} % Load characters and hyphenation
\fi
\usepackage[english=british]{csquotes}  % English quotes

% Load packages for testing
\usepackage{blindtext}
%\usepackage{showframe} % Uncomment to show boxes around the text area, margin, header and footer
%\usepackage{showlabels} % Uncomment to output the content of \label commands to the document where they are used

% Load the bibliography package
$if(bibliography)$
\usepackage{kaobiblio}
\addbibresource{$bibliography$} % Bibliography file
$endif$

% Load mathematical packages for theorems and related environments
\usepackage[framed=true]{kaotheorems}

% Load the package for hyperreferences
\usepackage{kaorefs}

\makeindex[columns=3, title=Alphabetical Index, intoc] % Make LaTeX produce the files required to compile the index

\makeglossaries % Make LaTeX produce the files required to compile the glossary
\input{glossary.tex} % Include the glossary definitions

\makenomenclature % Make LaTeX produce the files required to compile the nomenclature

% Reset sidenote counter at chapters
%\counterwithin*{sidenote}{chapter}

$if(tocTitle)$
\renewcaptionname{english}{\contentsname}{$tocTitle$}
$endif$
%----------------------------------------------------------------------------------------

$commonTemplate$

\begin{document}

%----------------------------------------------------------------------------------------
%  BOOK INFORMATION
%----------------------------------------------------------------------------------------

%\titlehead{The \texttt{kaobook} class}
%\subject{Use this document as a template}

$if(title)$
\title[$title$]{$title$}
%\title[Example and documentation of the {\normalfont\texttt{kaobook}} class]{Example and documentation \\ of the {\normalfont\texttt{kaobook}} class}
$endif$
%\subtitle{Customise this page according to your needs}

$if(author)$
\author[]{$for(author)$
    $author.name$$sep$, $endfor$}
$endif$
%\author[Federico Marotta]{Federico Marotta\thanks{A \LaTeX\ lover}}
\date{\today}

$if(book.publisher)$
\publishers{$book.publisher$}
%\publishers{An Awesome Publisher}
$endif$
%----------------------------------------------------------------------------------------

\frontmatter % Denotes the start of the pre-document content, uses roman numerals

%----------------------------------------------------------------------------------------
%  OPENING PAGE
%----------------------------------------------------------------------------------------

%\makeatletter
%\extratitle{
%  % In the title page, the title is vspaced by 9.5\baselineskip
%  \vspace*{9\baselineskip}
%  \vspace*{\parskip}
%  \begin{center}
%    % In the title page, \huge is set after the komafont for title
%    \usekomafont{title}\huge\@title
%  \end{center}
%}
%\makeatother

%----------------------------------------------------------------------------------------
%  COPYRIGHT PAGE
%----------------------------------------------------------------------------------------

\makeatletter
\uppertitleback{\@titlehead} % Header

\lowertitleback{
$if(disclaimer)$
  \textbf{Disclaimer}\\
$disclaimer$
%  You can edit this page to suit your needs. For instance, here we have a no copyright statement, a colophon and some other information. This page is based on the corresponding page of Ken Arroyo Ohori's thesis, with minimal changes.
$endif$

  \medskip

$if(copyright)$
$copyright$
%  \textbf{No copyright}\\
%  \cczero\ This book is released into the public domain using the CC0 code. To the extent possible under law, I waive all copyright and related or neighbouring rights to this work.
%
%  To view a copy of the CC0 code, visit: \\\url{http://creativecommons.org/publicdomain/zero/1.0/}
$endif$

  \medskip

$if(colophon)$
  \textbf{Colophon} \\
$colophon$
%  This document was typeset with the help of \href{https://sourceforge.net/projects/koma-script/}{\KOMAScript} and \href{https://www.latex-project.org/}{\LaTeX} using the \href{https://github.com/fmarotta/kaobook/}{kaobook} class.
%
%  The source code of this book is available at:\\\url{https://github.com/fmarotta/kaobook}
%
%  (You are welcome to contribute!)
$endif$

  \medskip

  \textbf{Publisher} \\
  First printed $if(book.firstprinted)$ in $book.firstprinted$ $endif$ $if(book.publisher)$ by \@publishers $endif$
}
\makeatother

%----------------------------------------------------------------------------------------
%  DEDICATION
%----------------------------------------------------------------------------------------

$if(dedicatory)$
\dedication{$dedicatory$
%  The harmony of the world is made manifest in Form and Number, and the heart and soul and all the poetry of Natural Philosophy are embodied in the concept of mathematical beauty.\\
%  \flushright -- D'Arcy Wentworth Thompson
}
$endif$

%----------------------------------------------------------------------------------------
%  OUTPUT TITLE PAGE AND PREVIOUS
%----------------------------------------------------------------------------------------

% Note that \maketitle outputs the pages before here

\maketitle

%----------------------------------------------------------------------------------------
%  PREFACE
%----------------------------------------------------------------------------------------

$if(preface)$
\chapter*{$preface-title$}
\addcontentsline{toc}{chapter}{$preface-title$}

$preface$
%\input{chapters/preface.tex}
\index{preface}
$endif$

%----------------------------------------------------------------------------------------
%  TABLE OF CONTENTS & LIST OF FIGURES/TABLES
%----------------------------------------------------------------------------------------

\begingroup % Local scope for the following commands

% Define the style for the TOC, LOF, and LOT
%\setstretch{1} % Uncomment to modify line spacing in the ToC
%\hypersetup{linkcolor=blue} % Uncomment to set the colour of links in the ToC
\setlength{\textheight}{230\hscale} % Manually adjust the height of the ToC pages

% Turn on compatibility mode for the etoc package
\etocstandarddisplaystyle % "toc display" as if etoc was not loaded
\etocstandardlines % "toc lines" as if etoc was not loaded

\tableofcontents % Output the table of contents

\listoffigures % Output the list of figures

% Comment both of the following lines to have the LOF and the LOT on different pages
\let\cleardoublepage\bigskip
\let\clearpage\bigskip

\listoftables % Output the list of tables

\endgroup

%----------------------------------------------------------------------------------------
%  MAIN BODY
%----------------------------------------------------------------------------------------

\mainmatter % Denotes the start of the main document content, resets page numbering and uses arabic numbers
\setchapterstyle{kao} % Choose the default chapter heading style

$body$

$if(appendix)$
\appendix % From here onwards, chapters are numbered with letters, as is the appendix convention
\pagelayout{wide} % No margins
\addpart{Appendix}
\pagelayout{margin} % Restore margins

$appendix$
$endif$
%----------------------------------------------------------------------------------------

\backmatter % Denotes the end of the main document content
\setchapterstyle{plain} % Output plain chapters from this point onwards

%----------------------------------------------------------------------------------------
%  BIBLIOGRAPHY
%----------------------------------------------------------------------------------------

% The bibliography needs to be compiled with biber using your LaTeX editor, or on the command line with 'biber main' from the template directory
$if(bibliography)$
\defbibnote{bibnote}{Here are the references in citation order.\par\bigskip} % Prepend this text to the bibliography
\printbibliography[heading=bibintoc, title=Bibliography, prenote=bibnote] % Add the bibliography heading to the ToC, set the title of the bibliography and output the bibliography note
$endif$

%----------------------------------------------------------------------------------------
%  NOMENCLATURE
%----------------------------------------------------------------------------------------

% The nomenclature needs to be compiled on the command line with 'makeindex main.nlo -s nomencl.ist -o main.nls' from the template directory
$if(constants)$

$if(constants.title)$\renewcommand{\nomname}{$constants.title$} % Rename the default 'Nomenclature'
$endif$
$if(constants.preamble)$\renewcommand{\nompreamble}{$constants.preamble$} % Prepend this text to the nomenclature
$endif$

$for(constants.content)$
\nomenclature{$constants.content.symbol$}{$constants.content.shortdef$: $constants.content.definition$ ($constants.content.value$)}
$endfor$
%\nomenclature{c}{Speed of light in a vacuum inertial frame}
%\nomenclature{h}{Planck constant}
%

\printnomenclature % Output the nomenclature
$endif$

%----------------------------------------------------------------------------------------
%  GREEK ALPHABET
%   Originally from https://gitlab.com/jim.hefferon/linear-algebra
%----------------------------------------------------------------------------------------

\vspace{1cm}

{\usekomafont{chapter}Greek Letters with Pronunciations} \\[2ex]
\begin{center}
  \newcommand{\pronounced}[1]{\hspace*{.2em}\small\textit{#1}}
  \begin{tabular}{l l @{\hspace*{3em}} l l}
    \toprule
    Character & Name & Character & Name \\
    \midrule
    alpha & alpha \pronounced{AL-fuh} & \nu & nu \pronounced{NEW} \\
    beta & beta \pronounced{BAY-tuh} & \xi, \Xi & xi \pronounced{KSIGH} \\
    gamma, \Gamma & gamma \pronounced{GAM-muh} & o & omicron \pronounced{OM-uh-CRON} \\
    delta, \Delta & delta \pronounced{DEL-tuh} & \pi, \Pi & pi \pronounced{PIE} \\
    epsilon & epsilon \pronounced{EP-suh-lon} & \rho & rho \pronounced{ROW} \\
    zeta & zeta \pronounced{ZAY-tuh} & \sigma, \Sigma & sigma \pronounced{SIG-muh} \\
    eta & eta \pronounced{AY-tuh} & \tau & tau \pronounced{TOW (as in cow)} \\
    theta, \Theta & theta \pronounced{THAY-tuh} & \upsilon, \Upsilon & upsilon \pronounced{OOP-suh-LON} \\
    iota & iota \pronounced{eye-OH-tuh} & \phi, \Phi & phi \pronounced{FEE, or FI (as in hi)} \\
    kappa & kappa \pronounced{KAP-uh} & \chi & chi \pronounced{KI (as in hi)} \\
    lambda, \Lambda & lambda \pronounced{LAM-duh} & \psi, \Psi & psi \pronounced{SIGH, or PSIGH} \\
    mu & mu \pronounced{MEW} & \omega, \Omega & omega \pronounced{oh-MAY-guh} \\
    \bottomrule
  \end{tabular} \\[1.5ex]
  Capitals shown are the ones that differ from Roman capitals.
\end{center}

%----------------------------------------------------------------------------------------
%  GLOSSARY
%----------------------------------------------------------------------------------------

% The glossary needs to be compiled on the command line with 'makeglossaries main' from the template directory

\setglossarystyle{listgroup} % Set the style of the glossary (see https://en.wikibooks.org/wiki/LaTeX/Glossary for a reference)
\printglossary[title=Special Terms, toctitle=List of Terms] % Output the glossary, 'title' is the chapter heading for the glossary, toctitle is the table of contents heading

%----------------------------------------------------------------------------------------
%  INDEX
%----------------------------------------------------------------------------------------

% The index needs to be compiled on the command line with 'makeindex main' from the template directory

\printindex % Output the index

%----------------------------------------------------------------------------------------
%  BACK COVER
%----------------------------------------------------------------------------------------

% If you have a PDF/image file that you want to use as a back cover, uncomment the following lines

%\clearpage
%\thispagestyle{empty}
%\null%
%\clearpage
%\includepdf{cover-back.pdf}

%----------------------------------------------------------------------------------------

\end{document}
